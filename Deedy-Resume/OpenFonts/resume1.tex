%%%%%%%%%%%%%%%%%%%%%%%%%%%%%%%%%%%%%%%
% Deedy - One Page Two Column Resume
% LaTeX Template
% Version 1.2 (16/9/2014)
%
% Original author:
% Debarghya Das (http://debarghyadas.com)
%
% Original repository:
% https://github.com/deedydas/Deedy-Resume
%
% IMPORTANT: THIS TEMPLATE NEEDS TO BE COMPILED WITH XeLaTeX
%
% This template uses several fonts not included with Windows/Linux by
% default. If you get compilation errors saying a font is missing, find the line
% on which the font is used and either change it to a font included with your
% operating system or comment the line out to use the default font.
% 
%%%%%%%%%%%%%%%%%%%%%%%%%%%%%%%%%%%%%%
% 
% TODO:
% 1. Integrate biber/bibtex for article citation under publications.
% 2. Figure out a smoother way for the document to flow onto the next page.
% 3. Add styling information for a "Projects/Hacks" section.
% 4. Add location/address information
% 5. Merge OpenFont and MacFonts as a single sty with options.
% 
%%%%%%%%%%%%%%%%%%%%%%%%%%%%%%%%%%%%%%
%
% CHANGELOG:
% v1.1:
% 1. Fixed several compilation bugs with \renewcommand
% 2. Got Open-source fonts (Windows/Linux support)
% 3. Added Last Updated
% 4. Move Title styling into .sty
% 5. Commented .sty file.
%
%%%%%%%%%%%%%%%%%%%%%%%%%%%%%%%%%%%%%%%
%
% Known Issues:
% 1. Overflows onto second page if any column's contents are more than the
% vertical limit
% 2. Hacky space on the first bullet point on the second column.
%
%%%%%%%%%%%%%%%%%%%%%%%%%%%%%%%%%%%%%%
\documentclass[]{deedy-resume-openfont}
\usepackage{fancyhdr}
 
\pagestyle{fancy}
\fancyhf{}
 
\begin{document}

%%%%%%%%%%%%%%%%%%%%%%%%%%%%%%%%%%%%%%
%
%     LAST UPDATED DATE
%
%%%%%%%%%%%%%%%%%%%%%%%%%%%%%%%%%%%%%%
\lastupdated

%%%%%%%%%%%%%%%%%%%%%%%%%%%%%%%%%%%%%%
%
%     TITLE NAME
%
%%%%%%%%%%%%%%%%%%%%%%%%%%%%%%%%%%%%%%
\namesection{Rahul}{Kumar}{ \urlstyle{same}\href{http://vernwalrahul.github.io}{vernwalrahul.github.io}\\
\href{mailto:vernwalrahul@iitkgp.ac.in}{vernwalrahul@iitkgp.ac.in} | +91 7261823455 
}

%%%%%%%%%%%%%%%%%%%%%%%%%%%%%%%%%%%%%%
%
%     COLUMN ONE
%
%%%%%%%%%%%%%%%%%%%%%%%%%%%%%%%%%%%%%%

\begin{minipage}[t]{0.33\textwidth} 

%%%%%%%%%%%%%%%%%%%%%%%%%%%%%%%%%%%%%%
%     EDUCATION
%%%%%%%%%%%%%%%%%%%%%%%%%%%%%%%%%%%%%%

\section{Education} 

\subsection{IIT Kharagpur}
\descript{BS in Computer Science}
\location{2016 - Till Date}
\location{ CGPA: 9.16 / 10.0 }
\sectionsep

\subsection{DAV Kapildev}
\location{Grad. May 2016|  Ranchi, India}
\location{Grade : 95.4\%}
\sectionsep

%%%%%%%%%%%%%%%%%%%%%%%%%%%%%%%%%%%%%%
%     LINKS
%%%%%%%%%%%%%%%%%%%%%%%%%%%%%%%%%%%%%%

\section{Links} 
Github:// \href{https://github.com/vernwalrahul}{\bf vernwalrahul} \\
LinkedIn://  \href{https://www.linkedin.com/in/vernwalrahul}{\bf vernwalrahul} \\
Medium://  \href{https://medium.com/rahulvernwal}{\bf @rahulvernwal} \\

%%%%%%%%%%%%%%%%%%%%%%%%%%%%%%%%%%%%%%
%     COURSEWORK
%%%%%%%%%%%%%%%%%%%%%%%%%%%%%%%%%%%%%%

\section{Coursework}
Programming and Data Structures \\
Algorithms \\
Software Engineering \\
Database Management System \\
Compilers + Operating Systems\\
Artificial Intelligence (AI) \\
Machine Learning \\
Reinforcement Learning \\
Information Retrieval \\
Image Processing \\
\sectionsep

%%%%%%%%%%%%%%%%%%%%%%%%%%%%%%%%%%%%%%
%     SKILLS
%%%%%%%%%%%%%%%%%%%%%%%%%%%%%%%%%%%%%%

\section{Skills}
\location{Languages}
\textbullet{} C \textbullet{}   C++ \textbullet{} Python \textbullet{} SQL \textbullet{} Java \\
\textbullet{} Matlab \textbullet{} \LaTeX\ \\ 
\location{Libraries and Tools:}
\textbullet{} Tensorflow \textbullet{} OpenCV \textbullet{} ROS \\  \textbullet{} OMPL \textbullet{} Docker \textbullet{} Flask \\

%%%%%%%%%%%%%%%%%%%%%%%%%%%%%%%%%%%%%%
%     RESPONSIBILITIES
%%%%%%%%%%%%%%%%%%%%%%%%%%%%%%%%%%%%%%

\section{Responsibilities}
\textit{Instructor / Mentor} \\
\textbullet{} MIT-IIT Robotics Workshop \\
\textbullet{} IEEE Robotics Winter Workshop \\
\textbullet{} Kharagpur Winter of Code 2017 \\~\\
\textit{Executive Head} \\ 
\textbullet{} Code Club, IIT Kharagpur \\
\textbullet{} Kharagpur Open Source Society \\

%%%%%%%%%%%%%%%%%%%%%%%%%%%%%%%%%%%%%%
%
%     COLUMN TWO
%
%%%%%%%%%%%%%%%%%%%%%%%%%%%%%%%%%%%%%%

\end{minipage} 
\hfill
\begin{minipage}[t]{0.66\textwidth} 


%%%%%%%%%%%%%%%%%%%%%%%%%%%%%%%%%%%%%%
%     PUBLICATIONS
%%%%%%%%%%%%%%%%%%%%%%%%%%%%%%%%%%%%%%

\renewcommand\refname{PUBLICATIONS}
\nocite{*}
\bibliographystyle{abbrv}
\bibliography{publications}

%%%%%%%%%%%%%%%%%%%%%%%%%%%%%%%%%%%%%%
%     EXPERIENCE
%%%%%%%%%%%%%%%%%%%%%%%%%%%%%%%%%%%%%%

\section{Experience}
\runsubsection{Amazon Robotics}
\descript{| Software Engineer - Robotics Intern }
\location{May 2019 - July 2019 | Seattle, USA}
\begin{tightemize}
\item Built end to end Stack for hands free automation of box picking using UR10 (6DoF Robotic Arm).
\item Designed perception module to identify boxes from time of flight image. 
\item Integrated controller, motion planning and calibration modules.
\item Deployed entire stack to AWS code pipeline.
\end{tightemize}
\sectionsep

\runsubsection{Personal Robotics Lab}
\descript{| University of Washington }
\location{ Research Intern  \hfill Advisor: \href{https://goodrobot.ai/}{Prof. Siddhartha Srinivasa}}
\location{May 2018 - July 2018 | Seattle, USA}
Topic : Learning Sampling Methods for constrained space motion planning
\begin{tightemize}
\item Devised non uniform sampling strategies to bias sampling in bottleneck regions.
\item Devised algorithms to increase robustness of the generated graph.
\item Our algorithm outperformed state of the art method on a wide range of problems | Accepted at IRoS '19
\end{tightemize}
Working Areas - \textbf{Deep Learning, AutoEncoders, Constrained Space Problems}
\sectionsep

%%%%%%%%%%%%%%%%%%%%%%%%%%%%%%%%%%%%%%
%     PROJECTS
%%%%%%%%%%%%%%%%%%%%%%%%%%%%%%%%%%%%%%

\section{Projects}

\runsubsection{Kharagpur Robo-Soccer Research Lab}
\location{\\ AI Team Member \hfill Advisor : \href{http://www.facweb.iitkgp.ac.in/~jay/}{Prof. Jayanta Mukhopadhyay}}
\location{Jan 2017 – Present | IIT Kharagpur}
Objective : To build autonomous soccer playing robots
\begin{tightemize}
\item Integrated path planning and Finite State Machines (FSM) architecture for Robocup Small Size League.
\item Designed a simulator for robots using PyQT.
\item Worked on kalman filter to tackle noisy data from camera images.
\end{tightemize}
Research Areas - Multi-agent systems, motion planning, noise filters, robot soccer
\sectionsep

\runsubsection{Digital Legal Assistant}
\descript{\\ Open Soft 2019, General Championships, IIT Kharagpur}
\begin{tightemize}
\item Developed the stack to search for related cases and acts for a given natural language query.
\item Used page ranking algorithms on citation graphs to determine the ordering of results
and cases on over 50000 supreme court cases.
\end{tightemize}
\sectionsep

%%%%%%%%%%%%%%%%%%%%%%%%%%%%%%%%%%%%%%
%     AWARDS
%%%%%%%%%%%%%%%%%%%%%%%%%%%%%%%%%%%%%%

\section{Awards} 
\begin{tabular}{rll}
2019	     & Final Round Worldwide  & Game of Drones | \textbf{NIPS'19} with \textbf{Microsoft}\\
2019	     & Final Round National  & Smart India Hackathon\\
2018	     & \textbf{3\textsuperscript{rd}} in National  & \textbf{IBM Blockchain Hackathon}\\
2017     & Worldwide & RoboCup SSL | First Indian Team  \\
2016     & \textbf{All India Rank 9\textsuperscript{th}}  & KVPY Fellowship\\
2016     & \textbf{top 0.03\%} (AIR 266)  & JEE Advanced \\
\end{tabular}
\sectionsep

\end{minipage} 

\section{Other Projects}

\runsubsection{Learning a robust walk engine for Nao robots}
\descript{\\ Jul'19 - Apr'20 \hfill Advisor : \href{http://www.facweb.iitkgp.ac.in/~jay/}{Prof. Jayanta Mukhopadhyay}} 
One of the major challenge in RoboCup Humanoid League is to enhance the speed and robustness of Nao walk engine. Together with my advisor,
I worked to build a walk engine for Nao Robotcs through Reinforcement Learning. We evaluated various different algorithms like evolution strategies, PPO, DDPG, and Soft Actor Critic Method. \\
Working Areas: \textbf{Reinforcement Learning, Evolution Strategies, Imitation Learning}.
\sectionsep

\runsubsection{Action/Event Recognition for Safety Analytics}
\descript{\\ Dec'17 - Feb'18 \hfill Advisor : \href{https://cse.iitkgp.ac.in/~pabitra/}{Prof. Pabitra Mitra}} 
Recognising actions in video clips by extending CNN in the time domain. The model developed to be most suited foran
industrial setting like detecting accidents in a factory. \\
Working Areas: \textbf{Computer Vision, ConvNets, Encoder Decoder Models}
\sectionsep

\runsubsection{Question Generation from RDF Graph via Discriminative Ranking}
\descript{\\ Aug'18 - Nov'18 \hfill Advisor : \href{http://www.iitkgp.ac.in/department/ET/faculty/et-plaban}{Prof. Plaban Bhowmick}} 
Developed an application to automatically generate Q/A pairs from RDF graphs. It involves identification of popular-
entities, extraction of their relation with other entities using hop distance. Extracted tokens are then fed to tranforma-
tions and ranking algorithm to produce a ranked list of questions. \\
Working Areas / Libraries: Knowledge Graph, Ranking Algorithm, SPARQL
\sectionsep

\runsubsection{Medical OCR}
\descript{\\ Jan'18 - Mar'18 } 
Worked in a team of 6 to build an OCR for detecting of medical professionals from prescriptions. Integrated Peter
Norvig’s spelling corrector algorithm to auto-correct misspelled words. \\
Working Areas: Computer Vision, Character Recognition, Spelling Correction
\sectionsep

\runsubsection{RRT Simulator}
\descript{\\ Repository: \href{https://github.com/vernwalrahul/RRTSimulator/}{RRTSimulator} } 
Developed an interactive GUI interface to simulate a path generated by RRTs avoiding obstacles using Python and Qt.
Added Features for low level skill testing of individual robots.
Tools and Libraries: OMPL, PyQt, ROS.
\sectionsep

\runsubsection{Blockchain Certificates}
\\ An application on digital certificates using blockchain technology to avoid fraud certificates and speed up the verifica-
tion process. \\
Won 3rd prize at National Level Hackathon.
\sectionsep

\section{Technical Blogs}
\descript{\hfill}
\begin{tightemize}
\item \href{https://tutorials.botsfloor.com/creating-your-messenger-bot-4f71af99d26b}{Creating Your Messenger Bot with Python} \hfill \textbf{21k views}
\item \href{https://becominghuman.ai/how-should-i-start-with-cnn-c62a3a89493b}{How Should I Start with CNN} \hfill 2.5k views
\item \href{https://medium.com/@rahulvernwal/an-introduction-to-vae-c347aea57849}{An Introduction to Variational Auto-Encoder} \hfill 1.1k views
\end{tightemize}
\sectionsep

\end{document}  \documentclass[]{article}
