\documentclass[a4paper,8pt]{article}
%-----------------------------------------------------------
\usepackage[top=0.25in, bottom=0.25in, left=0.55in, right=0.85in]{geometry}
\usepackage{graphicx}
\usepackage{url}
\usepackage{palatino}
\usepackage{tabularx}
\usepackage{hyperref}
\fontfamily{SansSerif}
\selectfont

\usepackage[T1]{fontenc}
\usepackage
%[ansinew]
[utf8]
{inputenc}

\usepackage{color}
\definecolor{mygrey}{gray}{0.75}
\textheight=10.75in
\raggedbottom

\setlength{\tabcolsep}{0in}
\newcommand{\isep}{-2 pt}
\newcommand{\lsep}{-0.5cm}
\newcommand{\psep}{-0.6cm}
\renewcommand{\labelitemii}{$\circ$}

\pagestyle{empty}
%-----------------------------------------------------------
%Custom commands
\newcommand{\resitem}[1]{\item #1 \vspace{-2pt}}
\newcommand{\resheading}[1]{{\small \colorbox{mygrey}{\begin{minipage}{0.975\textwidth}{\textbf{#1 \vphantom{p\^{E}}}}\end{minipage}}}}
\newcommand{\ressubheading}[3]{
\begin{tabular*}{6.62in}{l @{\extracolsep{\fill}} r}
	\textsc{{\textbf{#1}}} & \textsc{\textit{[#2]}} \\
\end{tabular*}\vspace{-8pt}}
%-----------------------------------------------------------
\begin{document}
\begin{center}
\textbf{\Huge Rahul Kumar \\ }  
\indent \hfill \break
\indent D410, Nehru Hall, \hfill Email-id : \textbf{vernwalrahul@iitkgp.ac.in} \\
\indent Indian Institute of Tehnology, Kharagpur \hfill Mobile No.: \textbf{7261823455} \\
\indent West Bengal, INDIA - 721302  \hfill Github: {https://github.com/vernwalrahul/} \\
\end{center}

\hspace{0.5cm}\\[-0.2cm]
\resheading{\textbf{ACADEMIC DETAILS} }\\[\lsep]
\\ 
%\begin{table}[ht!]
%\begin{center}

\indent \begin{tabular}{ l @{\hskip 0.25in} l @{\hskip 0.25in} l @{\hskip 0.25in} l @{\hskip 0.25in} l }
\hline
\textbf{Education} & \textbf{Institute} & \textbf{Year} & \textbf{CPI / \%} \\
\hline
B. Tech:\\
{Computer Science and Engineering} & IIT Kharagpur  & 2016- Till date & \textbf{9.20 / 10} \\ \\
Intermediate & DAV Kpildev, Ranchi & 2014 - 2016 & \textbf{95.4 \%}\\
\hline
\end{tabular}
%\end{center}
%\end{table}
\\ \hfill \break \\
\resheading{\textbf{ PUBLICATIONS} }\\[\lsep]
\begin{itemize} 
\item \textbf{LEGO: Leveraging Experience in RoadmapGeneration for Sampling-Based Planning} \\
	by Rahul Kumar, Aditya Mandalika, Sanjiban Choudhury, Siddhartha Srinivasa \\ 
	\textit{in Intelligent Robots and Systems, IROS, 2019 IEEE/RSJ International Conference, (Macau, China). Nov 2019} \href{https://arxiv.org/abs/1907.09574} {PDF}

\end{itemize}

\hspace{0.5cm}\\[-0.2cm]
\resheading{\textbf{WORK EXPERIENCE} }\\[\lsep]

\begin{itemize}
\item \textbf{Amazon Robotics, Seattle} \\
\small{Software Engineering Intern} \hfill \small{May'19-July'19}  \\
\textit{Topic: } \textit{End to end system to demo box stacking using a robotic arm} \\
\small{Worked to build end to end stack for hands free automation of box picking using UR10 (6-DoF robotic arm).Designed perception module to identify boxes from time-of-flight image using clustering and plane fitting methods.Integrated controller, motion planning and calibration modules.} \\
\textit{Working Areas: } \small{Perception, Path Planning, State Machines, Robotic Automation and Controls} 

\item \textbf{University of Washington} \\
\small{Summer Intern, Personal Robotics Lab} \hfill \small{May'18-July'18}  \\
\textit{Topic: } \textit{Learning Sampling Methods for Robot Motion Planning} \\
\small{Devised non-uniform sampling strategies that favor sampling in bottleneck regions to accelerate
the planning process simultaneously maintaining its quality in smoothly changing environment.} \\
\textit{Research Areas: } \small{Deep Learning, Variational AutoEncoders, Graph Space Planning, Constrained Space Problems} \\
\href {https://goodrobot.ai/} {\textit{Advisor: Prof. Siddhartha Srinivasa}}

\item \textbf{Kharagpur RoboSoccer Students' Group,} \small{ IIT Kharagpur} \\
\small{Software Teamm Member} \hfill \small{Feb'17 - Present}\\
\textit{Objective: } \textit{To build Autonomous Soccer Playing Robots} \\
\small{Implemented path planning algorithms and Finite State Machines (FSM) Architecture for RoboCup Small Size League Robots, Designed a simulator for
robots using PyQt, Worked on Kalman Filter to tackle noisy data from Camera, enhancing World Model of the Game State.} \\
\textit{Research Areas: } \small{Multi-agent systems, motion planning, noise filters, robot soccer} \\
\href{http://www.facweb.iitkgp.ac.in/~jay/}{\textit{Advisor: Prof. Jayanta Mukhopadhyay}}
\end{itemize}

\hspace{0.5cm}\\[-0.2cm]
\resheading{\textbf{PROJECTS} }\\[\lsep]
\begin{itemize}

\item \textbf{Action/Event Recognition for Safety Analytics} \hfill \small{Dec'17 - Feb'18} \\
\href{https://cse.iitkgp.ac.in/~pabitra/}{\textit{Prof. Pabitra Mitra}} \\
\small{Recognising actions in video clips by extending CNN in the time domain.  The model developed to be most suited foran industrial setting like detecting accidents in a factory.}\\
\textit{Working Areas:} \small{ Computer Vision, ConvNets, Encoder Decoder Models }

\item \textbf{Digital Legal Assistant | Open Soft 2019, General Championships, IIT Kharagpur} \\
\small{Developed the stack to search for related cases and acts for a given natural language query. Database was made of around 50000 supreme court case. Used
page ranking algorithms on citation graphs to determine the ordering of results and cases.}

\item \textbf{RRT Simulator} \\
\small{Developed an interactive GUI interface to simulate a path generated by RRTs avoiding obstacles using Python and Qt. Added Features for low level skill testing of individual robots.}\\
\textit{Tools and Libraries:} \small{OMPL, PyQt, ROS.}\\
\textit{Repository:} \href{https://github.com/vernwalrahul/RRTSimulator/}{\small{https://github.com/vernwalrahul/RRTSimulator/}}

\item \textbf{Question Generation from RDF graphs via Discriminative Ranking} \hfill \small{Aug'18 - Nov'18} \\
\small{Developed an application to automatically generate Q/A pairs from RDF graphs.  It involves identification of popularentities, extraction of their relation with other entities using hop distance.  Extracted tokens are then fed to tranforma-tions and ranking algorithm to produce a ranked list of questions.}\\
\href{http://www.iitkgp.ac.in/department/ET/faculty/et-plaban}{\textit{Advisor: Prof. Plaban Bhowmik}}

\item \textbf{Medical OCR} \hspace{0.5cm} \\
\small{Worked in a team of 6 to build an OCR for detecting of medical professionals from prescriptions. Integrated Peter Norvig's spelling corrector algorithm to auto-correct misspelled words.} 
	
\item \textbf{Blockchain Certificates} (IBM Blockchain Hackathon) \\
\small{An application on digital certificates using blockchain technology to avoid fraud certificates and speed up the verification process.} 
\end{itemize}


\hspace{0.5cm}\\[-0.2cm]
\resheading{\textbf{ RESPONSIBILITIES } }\\[\lsep]
\begin{itemize}
\item \textbf{Instructor} (MIT-IIT Robotics Workshop) \\
\href{https://cse.iitkgp.ac.in/~sudeshna/}{\textit{Guide: Prof. Sudeshna Sarkar}} \hfill \textit{1st May'17 - 15th May'17} \\
\small{Conducted a fortnight long Robotics workshop for higher school students
Introduced basics of C, C++, Processing and their applications in the field of robotics} 

\item \textbf{Image Processing Mentor} (IEEE Robotics Winter Workshop) \hfill \textit{Dec'17} \\
\small{Conducted a week-long IP workshop for 1st and 2nd year undergraduates at IIT Kharagpur, teaching basic Image Processing using OpenCV and C++.} 

\item \textbf{Mentor, Kharagpur Winter of Code, 2017} (IIT Kharagpur) \\
\small{Mentored a couple of students in KWoC (organized by Kharagpur Open Source Society
) which is a 5-week long GSoC-styled programme for students who are new to open source software development. } 

\item \textbf{Executive Head, Kharagpur Open Source Society} (IIT Kharagpur) \hfill \textit{Feb'17 - Present} \\
\small{Organised the Git, GitHub and Linux workshop for getting participants started with open source software develop-ment, during Open Source Summit, Kshitij.} 

\item \textbf{Executive Head, Code Club} (IIT Kharagpur) \hfill \textit{Aug'17 - Present} \\
\small{Organised AI BootCamp to let participants dive into the theory of Machine Learning and its applications during Up.AISummit, 2018.} 

\end{itemize}

\hspace{0.5cm}\\[-0.2cm]
\resheading{\textbf{AWARDS AND ACHIEVEMENTS} }\\[\lsep]
\begin{itemize}

\item \noindent \textbf{Smart India Hackathon - 2019} \\
Was among the top 6 teams competing for the Smart India Hackathon Grand Finale 2019 (Software)

\item \noindent \textbf{KVPY 2016.} \\
Secured All India Rank 9th in one of the presitigious examination initiated by Department of Science and Technology, Government of India

\item \noindent \textbf{IIT JEE Advanced 2016.} \\
Secured All India Rank 266 in JEE Advanced 2016.

\item \noindent \textbf{Robocup 2017.} \\
First team from India to qualify for SSL, Robocup 2017 held in Japan, among top 24 teams across the globe.

\item \noindent \textbf{Best Fresher, Conquest, Kshitij 2017} \\
Participated in an Robotics Event at Kshitij 2017 , Asia’s largest techno-management fest. Awarded the best fresher at IIT Kharagpur.

\item \noindent \textbf{IBM Blockchain Hackathon, Kshitij 2018} \\
Secured 3rd position in the national level hackathon organized at Kshitij, 2018.

\end{itemize}

\hspace{0.5cm}\\[-0.2cm]
\resheading{\textbf{RELEVANT COURSES} }\\[\lsep]
\begin{itemize}
\item \noindent \textbf{Completed: } *Programming and Data Structures, *Algorithms - I, II, Formal Language and Automata Theory, *Software Engineering,
Probability and Statistics. *Computer Architecture, Matrix Algebra, Knowledge Modelling and Semantic Technologies, *Compilers, *DBMS

\item \noindent \textbf{Ongoing: } Artificial Intelligence, Reinforcement Learning, Image Processing \\
\textit{*: Theory + Lab}
\end{itemize}

\hspace{0.5cm}\\[-0.2cm]
\resheading{\textbf{TECHNICAL SKILLS} }\\[\lsep]
\begin{itemize}
\item \noindent \textbf{Languages} C, C++, Python, Matlab, {\LaTeX{}} \\
\textbf{Libraries and Tools} Tensorflow,OMPL , OpenCV, ROS, PyQt, Octave \\
\textbf{Field of Interest} Computer Vision, Path Planning, Machine Learning, Decentralised Systems. \\
\end{itemize}

\end{document}

